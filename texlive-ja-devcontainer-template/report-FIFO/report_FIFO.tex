\documentclass[uplatex, twocolumn,10pt]{jsarticle}

\usepackage[dvipdfmx]{graphicx}
\usepackage{latexsym}
\usepackage{bmpsize}
\usepackage{url}
\usepackage{comment}


\begin{document}

\title{\bf{\LARGE{MQPsについて調べたこと} \\ \Large{メッセージの順序保証と重複について}}}
\author{小関廉}
\date{}
\maketitle

\section{はじめに}

国内外向けにデータの発生を通知するための仕組みとして、
Message Queueing Protocols(以下MQPs)が庁内で注目を集めている。
MQPsの代表的なものとしては、MQTTやAMQPなどが挙げられる。
本報告書で扱うプロトコルとそのバージョンは以下のものとする。



\begin{itemize}
    \item MQTT v3.1.1 \cite{MQTTVers22:online}
    \item MQTT v5.0 \cite{MQTTVers5:online}
    \item AMQP 0.9.1 \cite{AMQPWork14:online}
    \item AMQP 1.0 \cite{amqpcore25:online}
\end{itemize}

MQTTは3系から5系は、機能追加やアップデートであり連続性があるが、
AMQPの0.9.1と1.0は概念レベルで大きく異なっている。

MQPsをデータの発生通知のために使うにあたって、重要になるのが
メッセージの到着順序と重複の有無である。本報告書では
複数のMQPsの異なるバージョンに対して、メッセージの到着順序と重複の有無についてまとめることを目的とする。
また、本報告書はプロトコルレベルでの機能に着目し、実装で盛り込まれている機能についてはできるだけ言及しないものとする。

\section{MQPsに存在する概念}

\subsection{メッセージの到達保証}
\label{section:qos}

MQPsではメッセージを運ぶ際に、そのメッセージが到達したかどうかのチェックの有無やレベルを決めることができる。
この概念はメッセージの到着順序について理解する上で必要な概念である。
メッセージの到達確認については、プロトコルによって異なったアプローチをしているため
それぞれについて説明する

\subsubsection{MQTT}

MQTTにはv3系とv5系が存在するが、メッセージの到達保証については同様のアプローチをしているため
まとめて説明する。MQTTではQoSという指標でメッセージの到達確認のレベルを決定している。
QoSは0から2までの3段階あり、以下のように位置付けられている。

\begin{itemize}
    \item QoS = 0 : 確認しない
    \item QoS = 1 : 少なくとも一度は届ける(重複する可能性あり)
    \item QoS = 2 : 一度だけ届ける(重複なし)
\end{itemize}

MQTTでは上記のQoSについてACKでの実装が規定されている。

\subsubsection{AMQP 0.9.1}

AMQP 0.9.1ではMQTTのようにプロトコルレイヤーで到達保証に関するレベルが定められている訳ではない。
AMQP 0.9.1ではメッセージの到達保証についてAcknowledgementsという仕組みを定めている。
\cite{AMQPWork14:online}ではAcknowledgementsについて以下の2つのモデルが許されている。
ここから、明示的にACKを返すモデルを採用した実装では送達の確認が可能になると解釈できる。

\begin{itemize}
    \item サーバがメッセージを送った時点でコンテンツを削除する
    \item クライアントが明示的にAcknowledgementsを返すことを強制する
\end{itemize}

\subsubsection{AMQP 1.0}

AMQP 1.0では、メッセージの到達保証については明示的に定められていない。
AMQP1.0ではメッセージについてsettledという概念を定義しており、
このパラメータを受け渡し変化させることで、到達保証が可能である。

\section{メッセージの到着順序}

\subsection{MQTT}

MQTTではメッセージをpublishした順序が基本的に保持されて、subscriberに届けられる。
しかし、QoSとsubscriber側のバッファリングによって、順序が入れ替わる場合がある。
MQTTではACKを返さずに受け取ることができるメッセージ数をinflight\_massageと呼称している。

\begin{description}
    \item[QoS=0の場合]\mbox{}\\
    基本的にはpublishした順序に配信されるが、到達確認を行わないため
    到着順序は入れ替わる可能性がある.\\
    
    \item[QoS$>0$ かつ inflight\_message$>1$]\mbox{}\\
    基本的にはpublishした順序で配信される。しかし、メッセージのバッファリングと到達確認の両方を行っているため
    再送処理を行った場合、順序が入れ替わる可能性がある。例えば、1,2,3,4の順でpublishしたものを受けとる場合を考える。
    この一連のメッセージの受信中に2の取得でメッセージロストが発生した場合、1,2,3,2,3,4のような受け取り順序になる可能性がある。\\

    \item[QoS$>1$ かつ inflight\_message$=1$]\mbox{}\\
    到達順序は保証される。なぜなら1つのメッセージの到達確認が完了するまでは次のメッセージの送信を行わないためである。
    1,2,3,4の順でpublishしたものを受け取る場合、1,2,3,3,4の順で受け取る可能性はある。
    
\end{description}

\subsection{AMQP0.9.1}

AMQP0.9.1の仕様書\cite{AMQPWork14:online}の[4.7 Content Ordering Guarantees]で、サーバーからconsumerまでの経路が一つであるならば、送信した順序でメッセージが到着しなくてはならないと規定されている。そのため、条件を整えることでメッセージの到着順序を保証することができる。

\subsection{AMQP 1.0}

AMQP1.0の仕様書\cite{amqpcore25:online}の[2.1.2 Communication Endpoints]で、2ノード間でのバイトストリームにおいてフレーム到着順序については規定されている。しかし、publisher-broker-subscriberという三者のモデルになった場合にpublisherが送信した順序をbrokerが保持してsubscriberに送るかどうかについての規定は見つけられない。そのため、メッセージの到着順序保証については実装依存になると思われる

\section{メッセージの重複}

MQTTでは\ref{section:qos}で紹介したQoSに従う。
AMQP0.9.1とAMQP1.0については実装に依存する。

\section{参考文献}

\bibliography{test}
\bibliographystyle{junsrt}

\end{document}